\section{Ansible}
Ansible can be used to provision servers. It does not have statefiles and is idempotent, meaning it won't make changes unless it has to.
\subsection{Infrastructure}
In a network of servers, one server is the \textbf{host}. The host can connect to other machines using SSH. On the host, playbooks can be written in \texttt{yaml} files. Run a playbook by using \texttt{ansible-playbook playbook.yaml}

\begin{lstlisting}[language=yaml]
- name: Example Playbook
  hosts: web
  become: true
  vars:
    packages:
      - nginx
      - curl
    enable_service: true
    secret_password: "{{ vault_password }}"
  roles:
    - myrole
  tasks:
    - name: Install packages
      apt:
        name: "{{ item }}"
        state: present
      loop: "{{ packages }}"
      notify: restart nginx
    - name: Configure app if enabled
      template:
        src: app.conf.j2
        dest: /etc/app.conf
      when: enable_service
      tags: config
  handlers:
    - name: restart nginx
      service:
        name: nginx
        state: restarted
\end{lstlisting}
\subsection{Vaults}
Vaults can be used to encrypt data: The file \texttt{vault.yaml} with the contents \texttt{vault\string_password: "mysecret"} can be encrypted using \texttt{ansible-vault encrypt vault.yml} and then included in a play: \texttt{ansible-playbook playbook.yml --ask-vault-pass}
To create a file, use \texttt{ansible-vault create foo.yaml}
\subsection{Collections, Roles \& Tags}

\textbf{Collections} are bundles of plugins, roles and modules. Install them using \texttt{ansible-galaxy collection install <name>}, or define a requirements.yaml to install multiple collections at once.
\textbf{Roles} are an abstraction above playbooks, allowing to reuse configuration steps: create a role using \texttt{ansible-galaxy init <name>}, then use a role like in the example above.
\textbf{Tags} can be used to execute a subset of tasks instead of the whole playbook. Run only specific tags by appending \texttt{--tags <name>} at the end of the ansible-playbook command. There are also two special commands: Tag \textit{always} runs every time, except when explicitly skipped: \texttt{--skip-tags=always}. Tag \textit{never} does not run unless specified with \texttt{--tags=never}
\subsection{Jinja2}
Jinja2 is the templating engine which is used by Ansible. It is used to generate configuration files.
% TODO: maybe add example for jinja2
