\section{Kubernetes (K8)}
\begin{mdframed}
	\textbf{K8 Objects:} Persistent entities which signal a intent (e.g. for something to be created on the cluster) \\
	\textbf{K8 Controller:} Tracks a Object and is responsible for bringing the current State closer to the desired State.
\end{mdframed}

\textbf{Pod:} Represents a process running on your cluster. Should contain one container, multiple are possible. \\
\textbf{Sidecars:} Sidecars are containers that run along the primary container in the same pod. Example use cases might be logging, security, data synchronization. \\
\textbf{Init Containers:} Similar to sidecars, but run and finish before app containers.
\textbf{Volume:} Assigns physical Storage to a Pod \\
\textbf{ReplicaSet:} Makes sure that a specified number of replica pods are running. In practice, deplomyments are used. \\
\textbf{Deployment:} Allows to manage one or multiple Pods. \\
\textbf{Service:} API Resource to expose logical set of Pods in the namespace. Acts as a load balancer (round-robin). \\
\textbf{Ingress:} Provides external Access to a Service. \\
\textbf{DaemonSet:} Ensures that all (or some) Nodes run a copy of a pod. This can be used for running supporting applications (logs, storage)
\begin{lstlisting}[language=yaml]
apiVersion: apps/v1
kind: Deployment
metadata:
  name: app
spec:
replicas: 3 # automatically deploys replicaset
selector:
  matchLabels:
    app: web
template:
  metadata: # pod
    labels:
      app: web
  spec:
    strategy:
      type: RollingUpdate # rolling update
      rollingUpdate:
        maxUnavailable: 1
        maxSurge: 1
    initContainers:
    - name: init
      image: busybox
      command: ['sh', '-c', 'sleep 10']
    containers: # container
    - name: web
      image: nginx
      ports:
      - containerPort: 80
    - name: sidecar
      image: busybox
      command: ['sh', '-c', 'while true; do sleep 30; done']
\end{lstlisting}

\subsection{Namespaces}
Used to separate resources. Only resources in same namespace can communicate directly.

% TODO: maybe resource limitation

\subsection{Rolling Updates}
Rolling updates can be used in order to ensure that enough pods are always running.
Rolling updates can be using maxUnavailable (maximum No. of Pods upgrading at the same time) and maxSurge (max No. of Pods allowed to run beyond specified No. of replica)

\begin{lstlisting}[language=yaml]
spec:
\end{lstlisting}

\subsection{Scheduling}
The \texttt{kube-scheduler} determines which nodes run which Pods. We can influence this decision process:

\begin{lstlisting}[language=yaml]
kind: Pod
spec:
  nodeSelector:
    disktype: ssd # this label needs to be in pod.spec
\end{lstlisting}

To evaluate if a Node is eligible to run a Pod, the following things are considered: Port availability, CPU \& Memory resources, available volumes, specified labels. Additionally, scoring is used to evaluate remaining nodes with criteria: pods of same service should be on diffrent nodes, nodes with few used resources are prioritized, node affinity. \\
\textit{Taints} can also be applied on nodes and pods, pods wont be deployed on nodes with matching taints. \textit{Tolerations} can be used to make exceptions to taints. Types of taints: \textit{NoSchedule, PreferNoSchedule, NoExecute}

% TODO: starting from 25, s.60: storage, secrets, configmap, probes

\subsection{Commands}
\textbf{Apply a manifest.yaml:} \texttt{kubectl {create|apply|replace} -f manifest.yaml} \\
\textbf{Connecting to a Pod:} \texttt{kubectl exec -it nginx-xxx -- sh}
\textbf{Undo rollout:} \texttt{kubectl rollout undo}
