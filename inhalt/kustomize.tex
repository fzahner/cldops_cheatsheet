\section{Kustomize}

The kustomize.yaml defines the resources and transformations. \textit{Transformers} transform your manifests by allowing common fields in the resources to be specified in one place.

In case you want multiple variations on a common resource, the files in \texttt{base/} can declare common elements and the files in \texttt{overlays/} will declare differences.

\textbf{Patches} allow you to change (patch) already set values.

Traditionally, when editing ConfigMaps or Secrets, the deployment did not change, therefore no pods were restarted. \textbf{Generators} allow to define values in the \texttt{kustomize.yaml} file, where changes get detected and pods automatically restarted.
\begin{lstlisting}[language=yaml]
# A sample kustomize.yaml file
apiVersion: kustomize.config.k8s.io/v1
kind: Kustomization
resources:
  - deployment.yaml
  - service.yaml
# Transformers:
namePrefix: dev-
namespace: my-namespace
commonLabels:
  app: my-app
# Patch:
patches:
- patch: |-
    - op: replace
      path: /metadata/name
      value: nginx-server
  target:
    kind: Service
    name: nginx-app
# Generator:
configMapGenerator:
  - name: app-config
    literals:
      - LOG_LEVEL=debug
# Adding component from component example below
components:
- ../../components/mysql
\end{lstlisting}

\subsection{Components}

Components encapsulate a set of modifications. Below is an example component which adds a mysql database. This component can then be included in overlays, base directories (see above) or other components.

\begin{lstlisting}[language=yaml]
# Sample component file. Path: components/mysql/kustomization.yaml
apiVersion: kustomize.config.k8s.io/v1alpha1
kind: Component # not Kustomization
resources: # resource files are also under /mysql/
- deployment.yaml
- service.yaml
\end{lstlisting}

\subsection{Commands}

\begin{lstlisting}[language=sh]
kustomize build <dir> # Renders manifests.
kubectl apply -k <dir> # Applies kustomization via kubectl.
\end{lstlisting}
