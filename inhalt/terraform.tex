\section{Terraform}
TF doesn't speak directly with an SDK, but rather Terraform -> Provider -> Client SDK. Diffrent providers enable diffrent platforms (AWS, Azure, Kubernetes, ...). A sample in HCL (Hashicorp Configuration Language):
\begin{lstlisting}[language=hcl]
variable "instance_type" {
  default = "t2.micro"
}
provider "aws" {
  region = "us-east-1"
}
resource "aws_instance" "web" {
  ami           = "ami-0c55b159cbfafe1f0"
  instance_type = var.instance_type
}
output "public_ip" {
  value = aws_instance.web.public_ip
}
\end{lstlisting}
To deploy infrastructure, write HCL in files like \texttt{main.tf}, then run \texttt{terraform init}, \texttt{terraform plan} to show changes that would be made, \texttt{terraform apply} to actually apply the changes. Use \texttt{terraform destroy} to delete all made changes.
\subsection{State}
Terraform stores state in the \texttt{terraform.tfstate} file. When working in teams, this state file also has to be shared as the terraform command relies on is validity. This could for example be done via an S3 Bucket.
