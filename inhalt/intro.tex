\section{General Definitions}
\textbf{Definitions:}
\begin{itemize}

	\item \textbf{Cloud Operations} is the practice of managing and optimizing cloud-based services and infrastructure.
	\item \textbf{GitOps}: Git-based infrastructure and application deployment; uses Git as single source of truth; enables CI/CD, automation, version control, and declarative configuration.

	\item \textbf{DevOps} combines development and operations; focuses on automation, collaboration, CI/CD, monitoring, and agile delivery.
\end{itemize}
\subsection{DevOps Cycle}
\textbf{Plan} (add Objectives and Requirements to Backlog), \textbf{Code} (add Code to Repo), \textbf{Build} (Pipelines runs on push, builds and unit tests software), \textbf{Test} (Build is deployed to staging environement, tested using E2E, load, accessibility tests), \textbf{Release} (snapshot of code is versioned, changes are documented), \textbf{Deploy} (release is installed onto production environement), \textbf{Operate} (application should run smoothly, issues are troubleshooted and documented, infrastructure is scaled), \textbf{Monitor} (Application Data is gathered and used for planning)
\textbf{Difference Between Continuous Delivery \& Continuous Deployment:} Deployment automatically pushes from staging to production, in Delivery this is manual. \\
\textbf{CD\&D Deployment Strategies:} \\
\textbf{Rolling Deployment:} Update infrastructure gradually, minimal downtime \\
\textbf{Blue-Green:} Two environments: Old and new versions respectively \\
\textbf{Canary:} Small user group tests first \\
\textbf{Feature Flag:} Deploy but activate later, can be toggled \\
\textbf{Dark Launching:} Rolling out a feature invisible for users, test its performance in the background
% TODO: maybe add pipeline commands (see w1)
