\section{GitLab Pipelines}
\begin{lstlisting}[language=yaml]
stages:
  - build
  - test
  - deploy
cache:
  paths:
    - .cache/
build:
  stage: build
  script:
    - echo "Building..."
    - mkdir -p artifacts && echo "artifact" > artifacts/output.txt
  artifacts:
    paths:
      - artifacts/
    expire_in: 1 hour
test:
  stage: test
  dependencies:
    - build
  script:
    - cat artifacts/output.txt
deploy_staging:
  stage: deploy
  environment:
    name: staging
    url: https://staging.example.com
    on_stop: stop_staging # Unstages env
  script:
    - cat k8.yaml | envsubst | kubectl apply -f -
  artifacts:
    expire_in: 1 hour
stop_staging:
  stage: deploy
  environment:
    name: staging
    action: stop
  script:
    - echo "Stopping staging"
\end{lstlisting}
\subsection{Environments}
Describe where the code gets deployed (e.g. Local, Integration, Testing, Staging, Production). Can be linked to a K8 cluster (needs to be set up via GitLab UI):

\subsection{Cache \& Artifacts}
\textbf{Cache:} Shared across \textit{jobs} in different \textbf{stages} and even \textbf{pipelines} (if configured). Used for speeding up tasks (e.g., restoring dependencies).
\textbf{Artifacts:} Shared only with \textit{downstream jobs} in later \textbf{stages}. Must be explicitly declared as \texttt{dependencies} to access in later jobs. Uploaded to GitLab and optionally downloadable.
\subsection{Push- vs. Pull-Based Deployments}
\textbf{Push-Based:} + Easy to use, + flexible deployment targets, - firewall needs to be opened, - pipeline needs to be adjusted for new environments \textbf{Pull-Based:} + no need for open firewall, + better scaling, - agent needs to be installed in every cluster

\subsection{GitLab Runner}
GitLab Runner executes jobs defined in \texttt{.gitlab-ci.yml}.
Types: \textbf{Shared} (managed by GitLab) vs. \textbf{Specific} (self-managed).
Can run in various executors: \texttt{shell}, \texttt{docker}, \texttt{kubernetes}, etc.
Tagged runners can restrict which jobs they pick up.
