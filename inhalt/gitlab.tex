\section{GitLab Piplines}
\begin{lstlisting}[language=yaml]
stages:
  - build
  - test
  - deploy
cache:
  paths:
    - .cache/
build:
  stage: build
  script:
    - echo "Building..."
    - mkdir -p artifacts && echo "artifact" > artifacts/output.txt
  artifacts:
    paths:
      - artifacts/
    expire_in: 1 hour
test:
  stage: test
  dependencies:
    - build
  script:
    - cat artifacts/output.txt
deploy_staging:
  stage: deploy
  environment:
    name: staging
    url: https://staging.example.com
    on_stop: stop_staging # Unstages env
  script:
    - cat k8.yaml | envsubst | kubectl apply -f -
  artifacts:
    expire_in: 1 hour
stop_staging:
  stage: deploy
  environment:
    name: staging
    action: stop
  script:
    - echo "Stopping staging"
\end{lstlisting}
\subsection{Environments}
Describe where the code gets deployed (e.g. Local, Integration, Testing, Staging, Production). Can be linked to a K8 cluster (needs to be set up via GitLab UI):
\subsection{Push- vs. Pull-Based Deployments}
\textbf{Push-Based:} + Easy to use, + flexible deployment targets, - firewall needs to be opened, - pipeline needs to be adjusted for new environements \textbf{Pull-Based:} + no need for open firewall, + better scaling, - agent needs to be installed in every cluster
% TODO: gitlab runner section
